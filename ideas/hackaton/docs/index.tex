\documentclass[a4paper]{article}

\usepackage[spanish]{babel}
\author{hackaton.online}
\title{Hackaton}
\date{\today}
\begin{document}
    \maketitle
    \tableofcontents
    \section[Idea]{Idea Principal}
    \subsection[Inicio]{Como surge la idea}
    \paragraph{}
    Recientemente he participado en una hackaton a nivel nacional
    en la cual, a dia de hoy (20/10/22), aún no sabemos el resultado 
    pero creo que será positivo.
    \paragraph{}
    Esta es mi primera toma de contacto con este tipo de concursos/eventos
    y he de decir que la esperiencia ha sido buena, pero me ha dejado con 
    muchas cosas que son muy mejorables:
    \begin{itemize}
        \item La organización ha sido bastante caótica, sin tener unas bases, 
        ni saber si habría que seguir con el proyecto o simplemente como entregar 
        la documentación obligatoria.
        \item Las charlas, hechas en Zoom, estaban mal planteadas y con poco control o moderación 
        dando cabida en los chats o actividades a ciertas personas poco profesionales
        que contestaban de forma vulgar a las propuestas echas por los ponentes.
        \item Las charlas, aparte de lo indicado anteriormente, debían ser didácticas 
        y formatibas, pero la mayoría de ellas eran simples actividades que no formaban ni
        instriuían, eran más "relleno" y con poca calidad y escasas de contenido realmente 
        útil.
        \item La plataforma usada para gestionar los grupos iba algo al la mayoría de los 
        días, pareciendo una plataforma nueva sin mucho empeño
    \end{itemize}
    \subsection[¿Por que?]{El por qué de la idea}
    \paragraph{}
    Una vez vistos estos problemas, pienso que no sería tan difícil hacer una hackaton
    en condiciones, usando una plataforma propia sin depender de terceros (salvo quizás 
    para charlas) y dando documentación útil y formatiba, que se pueda consultar en 
    cualquier momento más que depender de asistir a unas charlas.
    \subsection[La Idea]{La idea como tal}
    \paragraph{}
    Con esta idea en mente y mis conocimientos actuales me gustaría hacer 
    una hackaton competente y útil, poniendo más enfoque en esto último y que 
    sea atractiva a los estudiantes y programadores permitiendo participar de 
    distintas formas, y no solo código como tal.

    \section[Prototipo]{Prototipo Documentado}
    Aquí paso a detallar el prototipo del proyecto, docmentado y explicando las partes, 
    ideas, fases y todo lo relacionado que pueda aclararse.
    \subsection[Web]{La Web}
    \subsubsection[/]{Inicio}
    \paragraph{}
    La web principal, es decir el Home, tendrá como parte principal, el logo.
    \paragraph{}
    Bajo el logo, está el número de participantes o, el tiempo que queda para abrir 
    la fase de inscripción.
    \paragraph{}
    Después de eso, enlaces directos a las bases (/bases) y a las instrucciones para 
    participar (/participo).Aparte, un último enlace a la idea del proyecto (/idea) 
    para darles algo de contexto sobre el proyecto y la idea.
    \paragraph{}
    Más abajo, los patrocinadores y personas que han participado\\ monetariamente 
    en ayudar e inpulsar el proyecto.
    \subsubsection[/idea]{Idea}
    \paragraph{}
    En esta parte, habrá básicamente una introducción igual o parecida a la 
    de esta documentación, ya que esta parte se trata de darle contexto al 
    participante de quienes somos y por qué surgió esta idea.
    \paragraph{}
    También habrá referencias a los patrocindores y como nos han ayudado a 
    desarrollar esta idea.
    \subsubsection[/bases]{Bases}
    \paragraph{}
    En esta parte, estarán explicadas claramente las reglas de 
    la hackaton:
    \begin{itemize}
        \item Cómo participar
        \item Que está permitido
        \item Que no se admite
        \item Material que se dará
        \item Tiempos de entrega
        \item Que se debe entregar y como 
        \item Beneficios de participar 
        \item Premios
        \item Requisitos mínimos
    \end{itemize}
    \subsubsection[/participo]{Participo}
    \paragraph{}
    Esta parte explicará de forma más visual y ejemplificada el como\\ se participaría, 
    desde las distintas modalidades hasta las formas "no code" de participar.
    \paragraph{}
    Dentro de esta sección, habrá ejemplos (/demo) muy visual, que son proyectos que por ejemplo 
    si serían totalmente validos, al igual que ejemplos que podrían parecerlo pero no lo serían.
    \subsubsection[/demo]{Demo}
    \paragraph{}
    Aquí habría dos partes: 
    \begin{itemize}
        \item 
        Un proyecto clave, que sería un proyecto de ejemplo que cumpliría con 
        todos los parametros.Este proyecto de prueba tendría tanto su parte visual, es decir, 
        la web o lo que vería como tal un usuario final, como la parte de documentación, 
        que tendría el mismo formato que el pedido a los participantes del hackaton.\\
        Esto último, la documentación, serviría a modo de guía para los nuevos participantes, 
        para que puedan saber en que formato se presenta y como sería la organización de los 
        documentos, así como una plantilla a seguir, aunque no al pie de la letra.\\
        ¡Innovar siempre se tendrá muy en cuenta!
        \item Un proyecto no representativo, el cual 
        será un proyecto que a primera vista parece apto, pero que no lo es y 
        se explicarán detalladamente aquellos puntos que no cumple y por que, así como
        cosas a evitar o faltas que pueden ser eliminatorias.
    \end{itemize}
    \subsubsection[/empresas]{Empresas}
    \paragraph{}
    Esta parte está enfocada a las empresas colaboradoras, que patrocinen o que quieran
    ayudar económicamente el proyecto. Aquí se listarán tanto los requisitos para patrocinar 
    el proyecto, cantidad mínima, formas de apoyar esta hackaton o incluso los beneficios de 
    participar con nosotros en esta hackaton.
    \subsubsection[más]{Más}
    \paragraph{}
    Aparte de estas partes, habría otras más que de primeras serían no se abrirían
    hast fases avanzadas o finales del proyecto.Estas serían (entre otras): 
    \begin{itemize}
        \item Panel del usuario
        \item Panel de notificaciones
        \item Zona de inspiraciones e ideas
        \item Zona de ideas x Empresa
        \item Zona de recursos
        \item Zona de logros
    \end{itemize}

    \subsection[Hackaton]{Hackaton}
    \paragraph{}
    Ahora vamos con la esencia de la hackaton, sus ideas principales
    que se plasmarán en las bases, como participar, cuándo, etc\dots
    \subsubsection[Concepto]{Concepto principal}
    \paragraph{}
    La hackaton se celebrará durante 5 días(y 2 más\\ opcionales), 
    teniendo dos entregas obligatorias, una al segundo día, de mera 
    documentación del proyecto (se tratará más adelante) y otra el 
    último día entregando el proyecto final. Una vez concluido ese 
    periodo (ordinario), existirán otros dos días que serían de 
    periodo extraordinario, donde se podrá actualizar la documentación 
    con apartados nuevos, pero no el proyecto final en sí.
    \paragraph{}
    Estos 5+2 días serían en un mes entre Enero y Marzo de forma totalmente
    online, respetando las fechas de exámene que puedan tener participantes que 
    estudien en la universidad o un grado de informática.
    \paragraph{}
    Estaría orientado a jovenes que están estudiando\\ alguna carrera,
    grado o incluso bachillerato que les guste la informática y 
    sepan desarrollar una web o una idea en condiciones. Como tal, no 
    hay límite de edad, pero si de experiencia.Solo se admitirá a 
    programadores Junior(menos de un año de experiencia)
    como participantes en el proyecto final,
    sin embargo, los programadores de más experiencia pueden 
    participar aportando ideas y llevandose un mérito especial y
    teniendo una parte dedicada del premio para ellos, sin quitarle 
    nada a sus compañeros menos expertos. También pueden hacer de 
    mentores en el grupo.
    \paragraph{}
    Una vez cerrado el proceso de inscripción, todos los participantes
    podrán acceder a su perfil y formar parte de un grupo, el cual puede 
    ser acordado (con sus compañeros, amigos, etc) o totalmente aleatorio
    participando con gente diversa y fomentando de estam forma 
    el trabajo en grupo y la relación con otros jovenes desarrolladores.
    \paragraph{}
    Cada grupo contará con un nombre de grupo, líder y "titular", que 
    será representativo del proyecto final.\\Los grupos se especifican al inscribirse.\\
    Cada grupo será de máximo 6 integrantes, sin mínimo de integrantes.Pueden 
    tener asignados hasta 2 mentores, o uno y un mentor ideal (esto se explicará más adelante)
    \paragraph{}
    En cuanto a mentores, estos pueden ser de cualquier perfil y edad y 
    pueden contribuir en la documentación pero no en el proyecto final.\\
    También existen los mentores ideales, que son los que han ofrecido 
    la idea al equipo de forma externa y es el equipo el que tiene 
    que desarrollarla. Estos, al igual que los anteriores, pueden ayudar 
    en la documentación pero no participar de forma activa en el proyecto 
    final.\\
    La principal diferencia entre un mentor "normal" y un mentor "ideal", 
    es que el mentor "normal" es alguien que está para dar consejos 
    basados en su experiencia al grupo, y guiarlos sobre ciertas decisiones.
    Mientras tanto, el mentor ideal es aquel que ha tenido la idea y que tiene 
    que expresarla y plasmarla para que el grupo la entienda y sea capaz 
    de desarrollarla.
    \paragraph{}
    Los grupos pueden indicar una breve descripción de su idea en el 
    momento del registro, aunque, en caso de que dejen en blanco esta 
    parte, se les dará a escoger entre ideas ya existentes en la 
    comunidad, ofrecidas por los mentores ideales, los cuales, si su 
    idea es escogida por algún equipo, formarán parte de el y deberán asesorar y 
    explicar su idea de forma más detallada al grupo para que este pueda llevarla a cabo.
    \paragraph{}
    La documentación a presentar será un PDF de una extensión menor a 20 hojas 
    si es solo texto y 30 hojas si incluye al menos 5 imágenes.\\
    Este PDF contendrá un indice de contenidos, un primer párrafo 
    introductorio indicando los participantes del grupo y un pequeño desglose de su 
    experiencia en el sector de la informática.\\
    Este documento debe indicar la idea en general, tecnologías a usar, que problema o 
    problemas soluciona y lo viable que es.\\
    Posteriormente, el documento debería contestar perfectamente a las siguientes preguntas
    (entre otras) :
    \begin{itemize}
        \item ¿Quiénes lo han hecho?
        \item ¿Por qué lo han hecho?
        \item ¿Cómo lo van a hacer?
        \item ¿Cómo afecta a la sociedad?
    \end{itemize}
\end{document}