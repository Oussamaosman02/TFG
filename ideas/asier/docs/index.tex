\documentclass[a4paper]{article}

\usepackage[spanish]{babel}

\title{ASIeR}
\author{Amalio}
\date{\today}
\begin{document}
\maketitle
\tableofcontents
\section[Idea]{Intraducción a la idea}
\paragraph{}
La idea tecnológica de este proyecto es construir una web,
que a la vez sea instalable en el móvil (PWA) con funciones de notificación,
base de datos y usando la librería de componentes de Material UI, inspirada en la filosofía
Material Design desarrollado por Google.
\paragraph{}
Aparte de esto, la idea general es construir una web/app
totalmente útil para difundir la información relevante de clase,
así como fechas de exámenes y tareas.
\section[Tecnologías]{Tecnologías a usar}
\subsection[Next Js]{NextJs(React)}
\paragraph{}
Esta web/app estará básicamente construida en\\ NextJs (12)
de Vercel, que trabaja sobre React (18)
creado y mantenido por Meta (Facebook).Este
framework nos permitirá ahorra tiempo de carga gracias a su
renderizado desde el servidor (SSR) y a su arquitectura de
una sola página (SPA) lo que dará dinamismo.
\subsection[Next-PWA]{next-pwa}
\paragraph{}
Next-Pwa es un módulo npm de javascript específico para NextJs que nos
permitirá crear la PWA de forma muy rápida, sencilla y sin mucha configuración extra.
\subsection[Web-Push]{web-push}
\paragraph{}
Este también es un módulo de javascript enfocado a manejar las notificaciones emergentes o "push"
que envía la web al dispositivo.Gracias a este módulo, conseguiremos eso usando la api interna del
navegador, sin tener que usar ningún servicio externo.
\subsection[MongoDB]{MongoDB}
\paragraph{}
También hemos comentado que usaremos una base de datos, en este caso es MongoDB aunque
no se descartaría usar una que sea SQL como Postgres (Supabase).
\subsection[Mongoose]{Mongoose}
\paragraph{}
Otro módulo npm que sirve para manejar MongoDB con sus drivers sin tener que hacerlo
nosotros mismos y así simplificarlo todo.
\subsection[MUI]{Materila UI}
\paragraph{}
Esta librería de interfaces de usuario para React,\\ basada en la idea del Material Design
dará una apariencia similar a las apps de google y por tanto más familiar y conocida.
\section[Explicación de funciones]{Explicación de funciones}
\subsection[Calendario]{Calendario con alertas}
\paragraph{}
Para el calendario, tendrá en las fechas que haya examen, un icono arriba a la derecha
y si hay una tarea habrá otro icono distinto para las tareas. Aparte, al clicar en la fecha
nos listará las tareas y examenes que hubiera. Aparte, si para el examen quedan menos de dos
días, esa fecha saldrá en el calendario con un fondo rojo y en el caso de que quede menos de
un día para una tarea su fondo será naranja.
\subsection[Tareas]{Tareas}
\paragraph{}
Las tareas irán con una descripción corta y un título, aparte indicando la asignatura a la
que pertenecen y la fecha límite.

\subparagraph{}
Aparte, a la hora de visualizar las tareas, se podrán añadir comentarios, que irán listados
con el nombre del comentarista y el comentario como tal.
\paragraph{Función Adicional}
También las tareas podrían tener un botón para indicar que se ha hecho y así ver
el número de personas que las ha hecho, a modo de presión social para hacerla.

\subsection[Notificaciones]{Notificaciones}
\paragraph{}
Las notificaciones llegarán con unas vibraciones más continudas si es fecha de examen y más cortas si
es una tarea. Aparte, dentro de la notificación se indicará en el título si es tarea o exámen y
en el cuerpo o mensaje de la notificación se indicará cuando es dicha tarea o dicho exámen.

\subsection[Administradores]{Administradores}
\paragraph{}
Para los administradores habrá una página especial donde podrán mandar notificaciones o añadir 
tareas y exámenes a la app. Aparte, cada vez que añadan una tarea o un exámen se mandará una 
notificación a todos los subscritos.

\subsection[Generales]{Funciones generales}
\paragraph{borrado dinámico}
De las tareas y exámenes que ya han pasado, así no se llena la base de datos de 
archivos innecesarios.

\paragraph{reminder}
Añadir una notificación automática cuando quede menos de dos días para un examen o un día para
una tarea.

\paragraph{autenticación}
Para la autenticación general usaremos la base de datos de MongoDB, donde los usarios tendrán un nombre 
y una contraseña específica con ese nombre.También, podría considerar usar Auth0, una librería específica
para ello.\\ Aparte, el nombre se guardará en el almacenamiento local para no tener que llamar todo el rato
a la base da datos y así evitar consumir recursos.
\subparagraph{}
Una vez "registrados", habrá un renderizado condicional, es decir, la página de inicio de sesión o registro 
solo se mostrará mientras no haya almacenado en memoria un nombre, en cuanto inicien sesión no volverá a 
aparecerles el login.

\paragraph{auth admins}
Para los administradores, debo usar un método disntinto de autenticación, ya que sus credenciales si 
deben ser 100\% secretas, por eso para ellos puede que si use Auth0 o les deje crear sus propios usuarios 
y contraseñas, pero siempre será un registro controlado por mi.Una idea sería crear una colección distinta
solo para ellos y crear una forma de añadirlos rápidamente con un simple form de dos inputs de tipo password. 
Entonces ellos (en mi presencia) irán añadiendo su usuario y contraseña de administrador enviandolo a la base 
de datos y una vez eso, ya estaría añadida dicha información, con una contraseña 
y usuario que solo ellos saben. 

\paragraph{Pupurrí de secciones }
Aquí dejo un listado de ideas que me parecen interesantes e implementables:\\
\begin{itemize}
    \item Sección de asignaturas con videos explicativos (yt)
    \item Sección de "resumenes" para cada exámen, por si alguien quiere aportar su resumen a la clase 
    \item Botón de "estoy en Discord" para avisar cuando alguien está en el discord de la clase
\end{itemize}
\end{document}