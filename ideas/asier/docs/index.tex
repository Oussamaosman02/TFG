\documentclass[a4paper]{article}

\usepackage[spanish]{babel}

\title{ASIeR}
\author{Amalio}
\date{\today}
\begin{document}
    \maketitle
    \tableofcontents
    \section[Idea]{Intraducción a la idea}
    \paragraph{}
    La idea tecnológica de este proyecto es construir una web,
    que a la vez sea instalable en el móvil (PWA) con funciones de notificación,
    base de datos y usando la librería de componentes de Material UI, inspirada en la filosofía
    Material Design desarrollado por Google.
    \paragraph{}
    Aparte de esto, la idea general es construir una web/app 
    totalmente útil para difundir la información relevante de clase, 
    así como fechas de exámenes y tareas.
    \section[Tecnologías]{Tecnologías a usar}
    \subsection[Next Js]{NextJs(React)}
    \paragraph{}
    Esta web/app estará básicamente construida en\\ NextJs (12)
     de Vercel, que trabaja sobre React (18) 
    creado y mantenido por Meta (Facebook).Este 
    framework nos permitirá ahorra tiempo de carga gracias a su
    renderizado desde el servidor (SSR) y a su arquitectura de 
    una sola página (SPA) lo que dará dinamismo.
    \subsection[Next-PWA]{next-pwa}
    \paragraph{}
    Next-Pwa es un módulo npm de javascript específico para NextJs que nos 
    permitirá crear la PWA de forma muy rápida, sencilla y sin mucha configuración extra.
    \subsection[Web-Push]{web-push}
    \paragraph{}
    Este también es un módulo de javascript enfocado a manejar las notificaciones emergentes o "push" 
    que envía la web al dispositivo.Gracias a este módulo, conseguiremos eso usando la api interna del 
    navegador, sin tener que usar ningún servicio externo.
    \subsection[MongoDB]{MongoDB}
    \paragraph{}
    También hemos comentado que usaremos una base de datos, en este caso es MongoDB aunque 
    no se descartaría usar una que sea SQL como Postgres (Supabase).
    \subsection[Mongoose]{Mongoose}
    \paragraph{}
    Otro módulo npm que sirve para manejar MongoDB con sus drivers sin tener que hacerlo 
    nosotros mismos y así simplificarlo todo.
    \subsection[MUI]{Materila UI}
    \paragraph{}
    Esta librería de interfaces de usuario para React,\\ basada en la idea del Material Design 
    dará una apariencia similar a las apps de google y por tanto más familiar y conocida.

\end{document}