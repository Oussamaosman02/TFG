\documentclass[a4paper]{article}

\usepackage[spanish]{babel}

\title{Hackaton}
\date{\today}
\begin{document}
    \maketitle
    \tableofcontents
    \section[Idea]{Idea Principal}
    \subsection[Inicio]{Como surge la idea}
    \paragraph{}
    Recientemente he participado en una hackaton a nivel nacional
    en la cual, a dia de hoy (20/10/22), aún no sabemos el resultado 
    pero creo que será positivo.
    \paragraph{}
    Esta es mi primera toma de contacto con este tipo de concursos/eventos
    y he de decir que la esperiencia ha sido buena, pero me ha dejado con 
    muchas cosas que son muy mejorables:
    \begin{itemize}
        \item La organización ha sido bastante caótica, sin tener unas bases, 
        ni saber si habría que seguir con el proyecto o simplemente como entregar 
        la documentación obligatoria.
        \item Las charlas, hechas en Zoom, estaban mal planteadas y con poco control o moderación 
        dando cabida en los chats o actividades a ciertas personas poco profesionales
        que contestaban de forma vulgar a las propuestas echas por los ponentes.
        \item Las charlas, aparte de lo indicado anteriormente, debían ser didácticas 
        y formatibas, pero la mayoría de ellas eran simples actividades que no formaban ni
        instriuían, eran más "relleno" y con poca calidad y escasas de contenido realmente 
        útil.
        \item La plataforma usada para gestionar los grupos iba algo al la mayoría de los 
        días, pareciendo una plataforma nueva sin mucho empeño
    \end{itemize}
    \subsection[¿Por que?]{El por qué de la idea}
    \paragraph{}
    Una vez vistos estos problemas, pienso que no sería tan difícil hacer una hackaton
    en condiciones, usando una plataforma propia sin depender de terceros (salvo quizás 
    para charlas) y dando documentación útil y formatiba, que se pueda consultar en 
    cualquier momento más que depender de asistir a unas charlas.
    \subsection[La Idea]{La idea como tal}
    \paragraph{}
    Con esta idea en mente y mis conocimientos actuales me gustaría hacer 
    una hackaton competente y útil, poniendo más enfoque en esto último y que 
    sea atractiva a los estudiantes y programadores permitiendo participar de 
    distintas formas, y no solo código como tal.

    \section[Prototipo]{Prototipo Documentado}
    Aquí paso a detallar el prototipo del proyecto, docmentado y explicando las partes, 
    ideas, fases y todo lo relacionado que pueda aclararse.
    \subsection[Web]{La Web}
    \subsubsection[/]{Inicio}
    \paragraph{}
    La web principal, es decir el Home, tendrá como parte principal, el logo.
    \paragraph{}
    Bajo el logo, está el número de participantes o, el tiempo que queda para abrir 
    la fase de inscripción.
    \paragraph{}
    Después de eso, enlaces directos a las bases (/bases) y a las instrucciones para 
    participar (/participo).Aparte, un último enlace a la idea del proyecto (/idea) 
    para darles algo de contexto sobre el proyecto y la idea.
    \paragraph{}
    Más abajo, los patrocinadores y personas que han participado monetariamente 
    en ayudar e inpulsar el proyecto.
    \subsubsection[/idea]{Idea}
    \paragraph{}
    En esta parte, habrá básicamente una introducción igual o parecida a la 
    de esta documentación, ya que esta parte se trata de darle contexto al 
    participante de quienes somos y por qué surgió esta idea.
    \paragraph{}
    También habrá referencias a los patrocindores y como nos han ayudado a 
    desarrollar esta idea.
    \subsubsection[/bases]{Bases}
    \paragraph{}
    En esta parte, estarán explicadas claramente las reglas de 
    la hackaton:
    \begin{itemize}
        \item Cómo participar
        \item Que está permitido
        \item Que no se admite
        \item Material que se dará
        \item Tiempos de entrega
        \item Que se debe entregar y como 
        \item Beneficios de participar 
        \item Premios
    \end{itemize}
\end{document}